\documentclass[a4paper,11pt]{article}

\usepackage[utf8]{inputenc}
\usepackage{polski}
\usepackage{hyperref}
\usepackage[vmargin=0.8in]{geometry}
\usepackage{listings}
\usepackage{xcolor}
\definecolor{backcolour}{rgb}{.9, .9, .9}
\lstset{backgroundcolor=\color{backcolour}, basicstyle=\ttfamily\footnotesize, breaklines=true}

\title{Zaawansowane Bazy Danych 2020\\
\large zadanie 1}

\author{Kamil Dubil\\
\href{mailto:kd370826@students.mimuw.edu.pl}{\tt kd370826@students.mimuw.edu.pl}}


\begin{document}

\maketitle

Wygenerowane dane umieściłem w katalogu \texttt{data/}.


\section{Wersja JSONowa}

Tabele z danymi:
\begin{lstlisting}
DROP TABLE IF EXISTS jsonb_audience, jsonb_targets;

CREATE TABLE jsonb_audience (
    date date NOT NULL,
    content jsonb NOT NULL
);
CREATE TABLE jsonb_targets (
    content jsonb NOT NULL
);
\end{lstlisting}
Wczytanie danych z plików:
\begin{lstlisting}
COPY jsonb_audience(content, date) FROM PROGRAM 'find /home/kd/zbd/zad1/data/ -type f -name "audience*json" -exec cat {} \; -exec echo -en "\t" \; -exec sh -c "basename {} | cut -c10-19" \;';
COPY jsonb_targets(content) FROM '/home/kd/zbd/zad1/data/targets.json';
\end{lstlisting}
Zapytanie:
\begin{lstlisting}
WITH audience AS (
    SELECT DISTINCT
        date,
        (jb->'person_id')::int person_id,
        jb->>'demography' demography,
        regexp_split_to_table(jb->>'contacts', '') contact
    FROM (
        SELECT
            date,
            jsonb_array_elements(content) jb
        FROM
            jsonb_audience
        ) ta
    WHERE
        jb->>'contacts' != ''
), targets AS (
    SELECT
        (jb->'target')::int id,
        replace(jb->>'definition', ' ', '_') definition
    FROM (
        SELECT
            jsonb_array_elements(content) jb
        FROM
            jsonb_targets
        ) tt
)
SELECT
    audience.date dzien,
    targets.id grupa,
    audience.contact reklama,
    count(*) osob
FROM
    audience
JOIN
    targets
ON
    audience.demography LIKE targets.definition
GROUP BY
    dzien,
    grupa,
    reklama
ORDER BY
    dzien,
    grupa,
    reklama;
\end{lstlisting}
Wynik:
\begin{lstlisting}
   dzien    | grupa | reklama | osob 
------------+-------+---------+------
 2019-01-01 |     0 | B       |    1
 2019-01-01 |     0 | C       |    2
 2019-01-01 |     0 | E       |    2
 ...
 2019-01-30 |     9 | X       |    1
 2019-01-30 |     9 | Y       |    2
 2019-01-30 |     9 | Z       |    3
(5721 rows)
\end{lstlisting}
Statystyki wykonania (fragment wyniku \texttt{EXPLAIN ANALYZE}):
\begin{lstlisting}
 Planning Time: 0.353 ms
 JIT:
   Functions: 23
   Options: Inlining true, Optimization true, Expressions true, Deforming true
   Timing: Generation 2.252 ms, Inlining 75.790 ms, Optimization 166.603 ms, Emission 108.365 ms, Total 353.010 ms
 Execution Time: 426.394 ms
\end{lstlisting}


\section{Wersja SQLowa}

Tymczasowe tabele z danymi:
\begin{lstlisting}
DROP TABLE IF EXISTS jsonb_audience, jsonb_targets;

CREATE TABLE jsonb_audience (
    date date NOT NULL,
    content jsonb NOT NULL
);
CREATE TABLE jsonb_targets (
    content jsonb NOT NULL
);
\end{lstlisting}
Wczytanie danych z plików:
\begin{lstlisting}
COPY jsonb_audience(content, date) FROM PROGRAM 'find /home/kd/zbd/zad1/data/ -type f -name "audience*json" -exec cat {} \; -exec echo -en "\t" \; -exec sh -c "basename {} | cut -c10-19" \;';
COPY jsonb_targets(content) FROM '/home/kd/zbd/zad1/data/targets.json';
\end{lstlisting}
Właściwe tabele i przeniesienie do nich danych:
\begin{lstlisting}
DROP TABLE IF EXISTS audience, targets;

CREATE TABLE audience (
    date date NOT NULL,
    person_id int NOT NULL,
    demography text NOT NULL,
    contact char NOT NULL,
    PRIMARY KEY (date, person_id, demography, contact)
);
CREATE TABLE targets (
    id int PRIMARY KEY,
    definition text NOT NULL
);

INSERT INTO audience (
    SELECT DISTINCT
        date,
        (jb->'person_id')::int person_id,
        jb->>'demography' demography,
        regexp_split_to_table(jb->>'contacts', '') contact
    FROM (
        SELECT
            date,
            jsonb_array_elements(content) jb
        FROM
            jsonb_audience
        ) ta
    WHERE
        jb->>'contacts' != ''
);

INSERT INTO targets (
    SELECT
        (jb->'target')::int id,
        replace(jb->>'definition', ' ', '_') definition
    FROM (
        SELECT
            jsonb_array_elements(content) jb
        FROM
            jsonb_targets
        ) tt
);

DROP TABLE IF EXISTS jsonb_audience, jsonb_targets;
\end{lstlisting}
Zapytanie:
\begin{lstlisting}
SELECT
    audience.date dzien,
    targets.id grupa,
    audience.contact reklama,
    count(*) osob
FROM
    audience
JOIN
    targets
ON
    audience.demography LIKE targets.definition
GROUP BY
    dzien,
    grupa,
    reklama
ORDER BY
    dzien,
    grupa,
    reklama;
\end{lstlisting}
Wynik:
\begin{lstlisting}
   dzien    | grupa | reklama | osob 
------------+-------+---------+------
 2019-01-01 |     0 | B       |    1
 2019-01-01 |     0 | C       |    2
 2019-01-01 |     0 | E       |    2
 ...
 2019-01-30 |     9 | X       |    1
 2019-01-30 |     9 | Y       |    2
 2019-01-30 |     9 | Z       |    3
(5721 rows)
\end{lstlisting}
Statystyki wykonania (fragment wyniku \texttt{EXPLAIN ANALYZE}):
\begin{lstlisting}
 Planning Time: 0.125 ms
 Execution Time: 29.539 ms
\end{lstlisting}


\section{Podsumowanie}

Wykonanie zapytania w wersji SQLowej jest istotnie szybsze. Jeśli doliczymy czas przenoszenia danych do właściwych tabel,
to wynik jest dużo gorszy, ale nadal lepszy od wersji JSONowej ---
\texttt{Execution Time} dla \texttt{INSERT INTO audience}: \texttt{288.620 ms}, dla \texttt{INSERT INTO targets}: \texttt{0.206 ms},
co łącznie daje \texttt{318.365 ms}.


\end{document}
